\section{5. Espectro de radiofrecuencias de Argentina}
El espectro de radiofrecuencias se encuentra dividido, seg\'un lo informado por la ENACOM, Ente Nacional de Comunicaciones,
en diferentes bandas atribuidas cada una a diferentes tipo de servicios que emplean el espectro electromagn\'etico para transmitir informaci\'on.

\begin{table}[H]
    \centering
    \begin{tabular}{c c}
        Servicios & Frecuencias \\
        Radiodifusi\'on AM & $535 - 1705 kHz$ \\
        Radiodifusi\'on FM & $88 - 108 MHz$ \\
        Radiodifusi\'on TV & $\textbf{VHF Bajo}: 54 - 72 MHz$ y $76 - 88 MHz$  $\textbf{VHF Alto}: 174 - 216 MHz$ $\textbf{UHF}: 512 - 806 MHz$ \\
        Telefon\'ia Celular & $\textbf{SRMC/STM}: 869 - 894 MHz$ y $824 - 849 MHz$ $\textbf{PCS}: 1850 - 1910 MHz$ y $1930 - 1990 MHz$ \\
    \end{tabular}
\end{table}

Se propuso utilizar una antena conectada al analizador de espectro para observar el espectro y lograr utilizar el analizar como demodulador de alguna se\~nal que no perteneciera
a las radios de AM, FM o TV, no obstante no se pudo conseguir dado que otro tipo de transmisiones son digitales y no se podr\'ia escuchar utilizando
el analizador de espectro para ello.
